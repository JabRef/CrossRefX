\documentclass[a4paper,10pt]{scrartcl}

\usepackage[utf8x]{inputenc}
\usepackage{textcomp}
\usepackage{xspace}
\usepackage{graphicx}
\usepackage[english]{babel}
\usepackage{microtype}
\usepackage{hyperref}
\usepackage{../commands_en}

\usepackage{xcolor}
\definecolor{darkblue}{rgb}{0,0,.5}
\definecolor{black}{rgb}{0,0,0}
\hypersetup{
	breaklinks=true,
	bookmarksnumbered=true,
	bookmarksopen=true,
	bookmarksopenlevel=1,
	breaklinks=true,
	colorlinks=true,
	pdfstartview=Fit,
	pdfpagelayout=SinglePage,
	%
	filecolor=darkblue,
	urlcolor=darkblue,
	linkcolor=black,
	citecolor=black
}


\title{User Manual to \crossrefx}
\author{}
\date{}
\begin{document}
\maketitle
\tableofcontents
\section{Introduction}
\crossrefx is a program to manage bibliographic references in the CrossTeX
format for LaTeX files. It is also possible to edit BibTeX-Files.
A detailed CrossTeX documentation can be found at
\url{http://www.cs.cornell.edu/people/egs/crosstex/index.php}.
The program by default uses an internal Apache Derby database, but it is also
possible to connect to a MySQL database.

Thanks to Mark James, FAMFAMFAM  \url{http://www.famfamfam.com/} for putting the
icons at the disposal by CC BY 3.0
\url{http://creativecommons.org/licenses/by/3.0/} license.
\section{Installation \& Startup}
\crossrefx is written entirely in Java, so for the program to work you will need
to have Java Runtime Environment
(\url{http://www.oracle.com/technetwork/java/javase/downloads/index.html})
installed correctly on your machine.

Having Java installed, to run the program you just have to execute
\texttt{crosstex-X.X.jar} in the program folder. Using Linux, starting in a
terminal could look like this: \texttt{java -jar \$DIR/crosstex-X.X.jar}.
\section{Quick-Start}
\begin{enumerate}
 \item Create new database: Click on the blank page
   \includegraphics{./images/application_add.png} icon on the toolbar.
 \item Create new entry: Click on \includegraphics{./images/report_add.png}
   and select the type of which your entry should created. A form will open and
   you can enter your data for the new entry.
 \item Export database to a xtx-file: Click on
   \includegraphics{./images/disk.png}. If you save this database the first
   time select a path to sore the xtx-file.
\end{enumerate}

\section{Functions}
At this point we want to give a short overview about what \crossrefx is capable
of.
\subsection{Create a new database}
To create a new database choose \clickpath{\file}{\newdatabase}.
In the new window select \internaldatabase, if you want to work locally without
a MySQL server. You have to set the name of the database. Click to \connectdb to
enter a database prefix.

If you choose a MySQL database in addition you have enter the IP adress of the
MySQL server to which you want to connect. To authenticate you enter your
username and your password. To create the new database click \ok. As a more
simple way you can create a new database also by clicking on the blank page
\includegraphics{./images/application_add.png} icon on the toolbar.
\subsection{Open a file}
To open a CrossTeX file, just click \clickpath{\file}{\openfile}.
Then you can either select an internal database or use a MySQL database if you
like. Enter the path to the file you want to open or click the \browse button
and select your file. To display all signs in the right way select the encoding
of your file. Click to \connectdb and select or enter a prefix. If you want to
work with a MySQL database, you additionally have to enter an IP-adress, port,
your username and your password.
Also you can simply click on this \includegraphics{./images/folder.png}
 symbol in the toolbar to open a file into a internal database.
 
\textbf{Please notice:} \textit{@default}-commands can be read, but all default
values will be written plaintext to the according spot. So reading and
afterwards saving a file with \textit{@default}-commands will erase them!
\subsection{Open a database}
To open a database which is created with \crossrefx choose
\clickpath{\file}{\openfile}. Enter the IP-adress of the database server and
your personal data. Click to \connectdb and select the prefix of the database
which you want to open.
\subsection{Export database to an CrossTeX-File}
To save a currently opened database into a CrossTeX file (.xtx) click
\clickpath{\file}{\exportdatabase} or on \includegraphics{./images/disk.png}
in the menubar. The database will be exported into the xtx-file from which the
database was opened or saved last. To save the database to a different path or
with a different filename you can choose \clickpath{\file}{\exportdatabaseas} or
\includegraphics{./images/disk_multiple.png}.
\subsection{Add a new entry}
To add a new entry to an opened database click \clickpath{\edit}{\newentry} or
on \includegraphics{./images/report_add.png} and select the type of which
your entry should created. A form will open and you can enter your data for the
new entry.
See Edit an entry (\ref{editentry}) to get to know how to change previously
created entries.
\subsection{Edit an entry}\label{editentry}
To edit a attribute of an entry, doubleclick on the entry a new subwindow will
open. There are two tabs shown, in the tab \requiredfields you can edit all
attributes which are necessary for this entry type.
In the second one, called  \optionalfields, you can edit all optional
attributes of the entry type. To set conditional options press the button
\includegraphics{./images/script.png} right of the textfield. In the new
subwindow you can the conditons and swayed attributetypes and values. Click
on \includegraphics{./images/add.png} to save the condition and add more if
you want to. To remove a conditon click on
\includegraphics{./images/delete.png}.

To link to an entry press button
\includegraphics{./images/table_relationship.png} right of the textfield. To
save you changes click \ok or \apply.

To every entry you can add a files into database. To do this click on
\optionalfields. In the first field click on
\includegraphics{./images/add.png} to add files to this entry. With
\includegraphics{./images/delete.png} you can delete them. To open a file
select it and click on \includegraphics{./images/folder.png}.

See also Undo and Redo (\ref{undoredo}).
\subsection{Edit includes}
To include other CossTex databases select \clickpath{\edit}{\IncludeText}. In
this dialog you can select one of the listed includes and edit them in the
\LabelText textfield or you can delete them with the \remove button. Enter the
name of an other CrossTex database to include it to your currently opened
database.
\subsection{Customize entry types}
In some cases new entry types are necessary or you might just want to customize
required and optional fields of an existing type. To do this select \edit
$\rightarrow$ \customizeentrytyp. The first list of the opened dialog lists all
types that are currently available in your document. In the second column you
see all required and in the third column all optional attributes of the selected
entry type.

New entry types you can add by entering a name in the textfield and press \add.
To delete a entry type just select it and press \remove.

To add required or optional types use the listbox below the corresponding
column and the \add button. With the up and down buttons you can order the
attributes as you like.
\subsection{Using the search}
To search entries with a specific content you can use the searchfield in the
toolbar. To start the search press the button
\includegraphics{./images/find.png} to right of the field or hit Enter.
The number of results will be shown right of the
\includegraphics{./images/resultset_previous.png} and
\includegraphics{./images/resultset_next.png} buttons. With the
\includegraphics{./images/resultset_previous.png} and
\includegraphics{./images/resultset_next.png} button you can navigate back
and forth through the given results.
\subsection{Undo \& Redo}
\label{undoredo}
These two buttons are made for reverting changes in the textfields (for example
while writing a new entry or editing an existing one). By pressing one of the
buttons once, a whole word is undone or redone.
\subsection{Edit an entry with plain-text editor} 
To edit a entry \"by hand\", select the entry. Right-click on it and choose
\editplain editor or click on \clickpath{\edit}{\editplain}. A
new subwindow with the CrossTex code will open. You can check the syntactical
correctness with the \checksyntax button. This will not add your changes to
database. To do this click on \checksave Click to \cancel quit this sub-window
without save anything.
\subsection{Show Objects \& Resolve Objects}
Here you can set up what happens to links between objects (this refers to the
object oriented behaviour, which is possible in CrossTeX).
If \showobjects is selected, such links will just be shown in the table, marked
with a \textrangle\textrangle prefix. Selecting \resolveobjects tries to follow
the link as far as possible and to show the resolved value in the table.
\section{Preferences}
Here you can change the language of \crossrefx or adjust the shortcuts
(\ref{shortcuts}) and the view (\ref{view}) as you like.
\subsection{Configuring language}
\label{language} 
To set the language to which you prefer, click on
\clickpath{\extras}{\clickpath {\preferences}{\generalpreferences}}. You can
select your language by the list. At moment English (EN), German (DE) and France
(FR) are supported. Feel free to add more languages to \crossrefx
\subsection{Configuring shortcuts}\label{shortcuts}
To configure a shortcut, you can choose a different modification key + some
number or letter from the listboxes. Of course setting two shortcuts to the same
value is not possible.
Also you should be aware that using the combination of \textit{Shift} + some
letter might cause some inappropriate behaviour.
\subsection{Configuring shown columns}\label{view}
To select the shown columns, click on
\clickpath{\extras}{\clickpath{\preferences}{\adjustview}}. Here you can select
more types from the combobox and add them with
\includegraphics{./images/add.png}. To remove types selct them and click on
\includegraphics{./images/delete.png}.
\end{document}
